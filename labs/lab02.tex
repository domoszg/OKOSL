\documentclass[12pt,a4paper]{article}
\usepackage[croatian]{babel}
\usepackage[utf8]{inputenc}
\usepackage[top=20mm]{geometry}
\usepackage{enumitem} 
\newcommand{\shell}[1]{\texttt{\textbf{#1}}}
\renewcommand*{\familydefault}{\sfdefault}
\renewcommand*{\sfdefault}{lmss}
\begin{document}
	\title{Laboratorijska vježba 2\\{\small Osnove korištenja operacijskog sustava Linux}\vspace{-2em}}
	\maketitle
	Za svaki zadatak potrebno je napisati jednu bash skriptu osim ako nije drugačije navedeno u tekstu zadatka. Kako biste lakše demonstrirali rješenja preporučamo da nakon svakog ključnog koraka u zadatku prikažete rezultate izvršavanja skripte. U tomu vam mogu pomoći sljedeće naredbe:
	\begin{description}[leftmargin=!,labelwidth=4em,itemsep=0em]
		\item[\shell{clear}] Čisti sadržaj terminala
		\item[\shell{read -p}] Čita podatak s tipkovnice; Zaustavlja izvođenje skripte
		\item[\shell{less}] Pager
	\end{description}
	
	\subsection*{Zadatak 1}
	Sljedeće zadatke je potrebno rješiti s datotekom \shell{/usr/include/stdio.h} koristeći regularne izraze (engl. \textit{regular expressions}).
	\begin{itemize}
		\item Prebrojite redove koji su uvučeni.
		\item Prebrojite koliko u datoteci postoji komentara i blokova s komentarima.
	\end{itemize}
	
	\subsection*{Zadatak 2}
	\begin{itemize}
		\item Ispisati sve datoteke čiji sadržaj počinje slovom \shell{c} iza kojeg slijede 2 slova, točka i barem jedno slovo iza točke \\ \textit{Primjer: \shell{cba.d} }
		\item Ispisati sve datoteke koje ne sadrže slova od \shell{a} do \shell{k}, ali imaju barem jednu znamenku.
	\end{itemize}
	
	\subsection*{Zadatak 3}
	Korištenjem \textit{here document} sintakse kreirati datoteku \shell{Top10} sadržaja:
	\begin{verbatim}
	Linux Mint 17.2
	Ubuntu 15.10
	Debian GNU/Linux 8.2
	Mageia 5
	Fedora 23
	openSUSE Leap 42.1
	Arch Linux
	CentOS 7.2-1511
	PCLinuxOS 2014.12
	Slackware Linux 14.1
	FreeBSD
	\end{verbatim}
	Nad datotekom je potrebno izvršiti sljedeće zadatke:
	\begin{itemize}
		\item Obrisati sve redove koji sadrže imena distribucija bez brojeva.
		\item Brojeve verzija prebaciti na početak reda.
		\item Sva slova promijeniti u mala.
		\item Zatim sve samoglasnike prebaciti u velika slova.
		\item Sortirati datoteku po numeričkoj vrijednosti brojeva na početku retka.
	\end{itemize}
	Ispišite sadržaj datoteke prije i poslije obrade.
  
	\subsection*{Zadatak 4}
	\begin{itemize}
		\item Pronaći sve Python datoteke na sustavu. Ispisati nazive svih funkcija iz njih.
		\item Pronaći sve C datoteke u sustavu. Ispisati sve pretprocesorske naredbe iz njih.
		\item Pronaći sve datoteke u sustavu koje sadrže niz \shell{include}. Ispisati broj retka u kojemu se nalazi pronađeni niz.
	\end{itemize}
  
	\subsection*{Zadatak 5}
	\begin{itemize}
    \item Datotekama koje imaju naziv oblika \shell{PNG-DDMMYYYY} (dan, mjesec, godina) promijeniti naziv u \shell{DD\_MM\_YYYY.png} \\ \textit{Primjer: \shell{PNG-07092015} $\longrightarrow$ \shell{07\_09\_2015.png}}
	\end{itemize}

	\subsection*{Zadatak 6}
  Kreirati datoteku kao i u Zadatku 3, sadrzaja:
	\begin{verbatim}
  <!DOCTYPE html>
  <html>
    <head>
      <title>Hi there</title>
    </head>
    <body>
      This is a page
      a simple page
    </body>
  </html>
	\end{verbatim}
	\begin{itemize}
    \item Ispisati sav sadrzaj izmedu HTML oznake \textbf{html}.
    \item Ispisati sav sadrzaj datoteke, bez HTML oznaka.
	\end{itemize}

\end{document}
