\documentclass[12pt,a4paper]{article}
\usepackage[croatian]{babel}
\usepackage[utf8]{inputenc}
\usepackage[top=20mm]{geometry}
\newcommand{\shell}[1]{\texttt{#1}}
\begin{document}
	\title{Laboratorijska vježba 2\\{\small Osnove korištenja operacijskog sustava Linux}\vspace{-2em}}
	\maketitle
	\subsection*{Zadatak 1}
	Kreirati datoteku \shell{Top10} sadržaja:
	\begin{verbatim}
	Linux Mint 17.2
	Ubuntu 15.10
	Debian GNU/Linux 8.1
	Mageia 5
	Fedora 22
	openSUSE 13.2
	Arch Linux
	CentOS 7.1-1503
	PCLinuxOS 2014.12
	Slackware Linux 14.1
	FreeBSD
	\end{verbatim}
	Sve točke zadatka se odnose na rad s tom datotekom. Kreiranje datoteke i upisivanje podataka u nju nisu dio zadatka.
	\begin{itemize}
		\item Obrisati sve redove koji sadrže imena distribucija bez brojeva.
		\item Brojeve verzija prebaciti na početak reda.
		\item Sva slova staviti u lower case.
		\item Zatim sve samoglasnike prebaciti u upper case.
		\item Tako dobivenu datoteku sortirati po veličini broja na početku reda.
		\item[] Napomena: \textit{Ne sortirati abecedno.}
	\end{itemize}
  
	\subsection*{Zadatak 2}
	Sve točke zadatka se odnose na rad s datotekama s vaseg cijelog sustava. Vase rjesenje mora funkcionirati bez obzira na postojanje pojedinih datoteka. Datoteke stvorene nisu dio rjesenja.
	\begin{itemize}
    \item Pronaci sve Python datoteke u sustavu te ispisati nazive funkcija iz njih. (Iskljucivo nazive funkcija)
		\item Pronaci sve C datoteke u sustavu te ispisati sve pretprocesorske naredbe iz njih.
    \item Pronaci sve datoteke u sustavu koje sadrze niz \textfb{include} i redak u kojem se on nalazi.
	\end{itemize}

	\subsection*{Zadatak 3}
	\begin{itemize}
    \item Ispisati sve datoteke koje pocinju slovom \texbf{c}, imaju 3 slova prije tocke, te barem jedno slovo nakon tocke. 
    \item Ispisati sve datoteke koje ne sadrze slova \texbf{a} do \text{k}, ali imaju barem 1 znamenku.
	\end{itemize}
  
	\subsection*{Zadatak 4}
	\begin{itemize}
    \item Promijeniti nazive datoteka koje imaju oblik PNG-07092015 (PNG dan mjesec godina) u 7_9_2015.png
	\end{itemize}

	\subsection*{Zadatak 5}
  Koristeci log datoteku rijesite slijedece zadatke.
	\begin{itemize}
    \item Ispisati HTTP statuse i broj pojavljivanja, sortirano po broju pojavljivanja.
    \item Ispisati ukupni broj prenesenih bajtova.
	\end{itemize}

  \subsection*{Zadatak 6}
	\begin{itemize}
    \item Sto oznacuje slovo \textbf{s} na lokaciji vlasnikove \textbf{executable} dozvole? \textit{-rwsr-xr-x 2 root root 32K May 22  2015 /usr/bin/su}
    \item Sto oznacuje dozvola \textbf{x} na direktoriju?
	\end{itemize}

\end{document}
